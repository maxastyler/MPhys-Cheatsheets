\documentclass[10pt,landscape]{article}
\usepackage{multicol}
\usepackage{calc}
\usepackage{ifthen}
\usepackage{amsmath}
\usepackage{tikz}
\usepackage[landscape]{geometry}
\usepackage{hyperref}
\usepackage{bm}
\usepackage{caption}
\captionsetup{
   font=small,
   labelfont=bf,
   tableposition=top
}
\usetikzlibrary{decorations.markings}
\tikzset{
    set arrow inside/.code={\pgfqkeys{/tikz/arrow inside}{#1}},
    set arrow inside={end/.initial=>, opt/.initial=},
    /pgf/decoration/Mark/.style={
        mark/.expanded=at position #1 with
        {
            \noexpand\arrow[\pgfkeysvalueof{/tikz/arrow inside/opt}]{\pgfkeysvalueof{/tikz/arrow inside/end}}
        }
    },
    arrow inside/.style 2 args={
        set arrow inside={#1},
        postaction={
            decorate,decoration={
                markings,Mark/.list={#2}
            }
        }
    },
}


% To make this come out properly in landscape mode, do one of the following
% 1.
%  pdflatex latexsheet.tex
%
% 2.
%  latex latexsheet.tex
%  dvips -P pdf  -t landscape latexsheet.dvi
%  ps2pdf latexsheet.ps


% If you're reading this, be prepared for confusion.  Making this was
% a learning experience for me, and it shows.  Much of the placement
% was hacked in; if you make it better, let me know...


% 2008-04
% Changed page margin code to use the geometry package. Also added code for
% conditional page margins, depending on paper size. Thanks to Uwe Ziegenhagen
% for the suggestions.

% 2006-08
% Made changes based on suggestions from Gene Cooperman. <gene at ccs.neu.edu>


% To Do:
% \listoffigures \listoftables
% \setcounter{secnumdepth}{0}


% This sets page margins to .5 inch if using letter paper, and to 1cm
% if using A4 paper. (This probably isn't strictly necessary.)
% If using another size paper, use default 1cm margins.
\ifthenelse{\lengthtest { \paperwidth = 11in}}
	{ \geometry{top=.5in,left=.5in,right=.5in,bottom=.5in} }
	{\ifthenelse{ \lengthtest{ \paperwidth = 297mm}}
		{\geometry{top=1cm,left=1cm,right=1cm,bottom=1cm} }
		{\geometry{top=1cm,left=1cm,right=1cm,bottom=1cm} }
	}

% Turn off header and footer
\pagestyle{empty}
 

% Redefine section commands to use less space
\makeatletter
\renewcommand{\section}{\@startsection{section}{1}{0mm}%
                                {-1ex plus -.5ex minus -.2ex}%
                                {0.5ex plus .2ex}%x
                                {\normalfont\large\bfseries}}
\renewcommand{\subsection}{\@startsection{subsection}{2}{0mm}%
                                {-1explus -.5ex minus -.2ex}%
                                {0.5ex plus .2ex}%
                                {\normalfont\normalsize\bfseries}}
\renewcommand{\subsubsection}{\@startsection{subsubsection}{3}{0mm}%
                                {-1ex plus -.5ex minus -.2ex}%
                                {1ex plus .2ex}%
                                {\normalfont\small\bfseries}}
\makeatother

% Define BibTeX command
\def\BibTeX{{\rm B\kern-.05em{\sc i\kern-.025em b}\kern-.08em
    T\kern-.1667em\lower.7ex\hbox{E}\kern-.125emX}}

% Don't print section numbers
\setcounter{secnumdepth}{0}


\setlength{\parindent}{0pt}
\setlength{\parskip}{0pt plus 0.5ex}


% -----------------------------------------------------------------------

\begin{document}

\raggedright
\footnotesize
\begin{multicols}{3}


% multicol parameters
% These lengths are set only within the two main columns
%\setlength{\columnseprule}{0.25pt}
\setlength{\premulticols}{1pt}
\setlength{\postmulticols}{1pt}
\setlength{\multicolsep}{1pt}
\setlength{\columnsep}{2pt}

\begin{center}
     \Large{Classical Electrodynamics Cheat Sheet} \\
\end{center}
\section{Maxwell's Equations}
\begin{align*}
	\underline \nabla \cdot \underline E &= \rho & \underline \nabla \cdot \underline B &= 0 \\
	\underline \nabla \times \underline E &= -\frac{1}{c}\frac{\partial \underline B}{\partial t} &
	\underline \nabla \times \underline B &= \frac{1}{c} \underline J + \frac{1}{c}\frac{\partial \underline E}{\partial t}
\end{align*}
With $\underline J=\rho \underline v$, the current density.
The force density is 
\begin{align*}
	\underline f = \rho \underline E + \frac{1}{c}\underline J \times \underline B
\end{align*}
Charge conservation is given by:
\begin{equation*}
	\frac{\partial \rho}{\partial t} + \underline \nabla \cdot \underline J = 0
\end{equation*}
And energy conservation by:
\begin{equation*}
	\frac{\partial (u+w)}{\partial t} + \underline \nabla \cdot \underline S = 0
\end{equation*}
Where $u=\frac{1}{2}(|\underline E|^2 + |\underline B|^2)$ is the electromagnetic charge density, $w=\int \underline f \cdot d\underline r$ is the work density and $\underline S=c\underline E\times\underline B$ is the Poynting vector, the flow of energy per unit time per unit area.

We can write the electric and magnetic fields in terms of a scalar potential and vector potential, $\phi, \underline A$.

\begin{align}
	\underline E &= -\underline\nabla\phi - \frac{1}{c}\frac{\partial \underline A}{\partial t} & \underline B &= \underline \nabla \times \underline A
\end{align}

These potentials are invariant up to a total divergence of a function $\chi$:
\begin{align}
	\phi &\rightarrow \phi-\frac{1}{c}\frac{\partial \chi}{\partial t} & \underline A &\rightarrow \underline A + \underline\nabla\chi
\end{align}
\section{Special Relativity}
In special relativity, there are 4-vectors and 4-vector duals, with bases:
\begin{align*}
	\bm{e}_{\mu}&=\frac{\partial}{\partial x^{\mu}} & \mathrm{and} && \bm{e}^{\nu}&=dx^{\nu}
\end{align*}
The bases have the property:
\begin{align*}
	\bm{e}_{\mu}(\bm{e}^{\nu}) = \bm{e}^{\nu}(\bm{e}_{\mu})= \delta_{\mu}^{\nu}
\end{align*}

There is a metric tensor $\bm{g}=g_{ij}\bm{e}^i\otimes\bm{e}^j$ which takes a 4-vector into its dual. In special relativity, the metric tensor is:
\begin{align*}
	\eta_{ij}=
	\begin{pmatrix}
		1 & 0 & 0 & 0 \\
		0 & -1 & 0 & 0 \\
		0 & 0 & -1 & 0 \\
		0 & 0 & 0 & -1
	\end{pmatrix}
\end{align*}
Distances in special relativity are given by:
\begin{align}
	ds^2=\eta_{ij}dx^idx^j
\end{align}
Where $dx^i = (cdt, dx, dy, dz)$ in pseudo-euclidean coordinates.
We can consider transformations which leave the length $ds$ invariant, Lorentz transformations:
\begin{align}
	\bm{\Lambda}\bm{\eta}\bm{\Lambda}^T=\bm{\eta}
\end{align}
In component form, a Lorentz boost in the $x$-direction can be written:
\begin{align}
	\Lambda^i{}_j =\begin{pmatrix}
		\gamma & -\gamma\beta & 0 & 0 \\
		-\gamma\beta & \gamma & 0 & 0 \\
		0 & 0 & 1 & 0 \\
		0 & 0 & 0 & 1 \\
	\end{pmatrix}
\end{align}
The totally anti-symmetric tensor is given by:
\begin{align}

\end{align}

\end{document}

\end{verbatim}

\rule{0.3\linewidth}{0.25pt}
\scriptsize


\end{multicols}
\end{document}
